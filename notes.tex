\documentclass[hidelinks]{article}[12pt]
% \documentclass{article}
% \usepackage{multirow,makeidx,algorithmic,algorithm}
\usepackage{multirow,makeidx,algpseudocode,algorithm}
\usepackage{mathtools}
\usepackage{enumitem}
\usepackage{booktabs}
 
\usepackage{parskip}
\usepackage{setspace}
\usepackage{float}
\usepackage{adjustbox}
\usepackage{bbm}
\usepackage{tabularx}
\usepackage{subfigure}
\usepackage{amsmath,amssymb,amsfonts,amsthm,amsbsy,amstext,mathrsfs}
\usepackage{microtype}

% \usepackage{hyperref}
\usepackage[colorlinks=true, % turn on colored links
            linkcolor=cyan,% internal links (sections, etc.)
            citecolor=blue, % citation links
            urlcolor=blue  % URL links
           ]{hyperref}
\usepackage{url}
\usepackage{color}
\usepackage{xcolor}
\usepackage{graphicx} % Required for inserting images
\usepackage[utf8]{inputenc}
\usepackage{appendix}


%% reference
\usepackage[round]{natbib}
\bibliographystyle{abbrvnat}
% \usepackage{cite}
% \bibliographystyle{plainurl}
% \bibliographystyle{abbrv}
% \bibliographystyle{plain}
% \bibliographystyle{unsrt}


%% code in-text
\usepackage{listings}
\usepackage{xcolor}

\definecolor{codegreen}{rgb}{0,0.6,0}
\definecolor{codegray}{rgb}{0.5,0.5,0.5}
\definecolor{codepurple}{rgb}{0.58,0,0.82}
\definecolor{backcolour}{rgb}{0.95,0.95,0.92}

\lstdefinestyle{mystyle}{
    backgroundcolor=\color{backcolour},   
    commentstyle=\color{codegreen},
    keywordstyle=\color{magenta},
    numberstyle=\tiny\color{codegray},
    stringstyle=\color{codepurple},
    basicstyle=\ttfamily\footnotesize,
    breakatwhitespace=false,         
    breaklines=true,                 
    captionpos=b,                    
    keepspaces=true,                 
    numbers=left,                    
    numbersep=5pt,                  
    showspaces=false,                
    showstringspaces=false,
    showtabs=false,                  
    tabsize=2
}

\lstset{style=mystyle}


%% layout
\oddsidemargin 3mm
\evensidemargin 3mm
\topmargin -12mm
\textheight 660pt
\textwidth 450pt


% \newcounter{xxx}
% \setcounter{xxx}{0}
% \newcommand\XXX[1]{{\bf \em \addtocounter{xxx}{1} (\thexxx) [[#1]]}}
% \newcommand{\revise}[1]{{\color{red} #1}}


\floatstyle{ruled}
\newfloat{function}{htbp}{lof}% “lof” = file extension for list
\floatname{function}{Function}% caption prefix
\newcommand{\listoffunctions}{\listof{function}{List of Functions}}
  % DO NOT CHANGE
\input{macros/basic-math-macros} 




\title{\bf Notes on Hamiltonian Monte Carlo}
\author{Junsong Tang}
\date{\today}

\setlength\parindent{0pt}
\frenchspacing
\begin{document}
\maketitle

\section{Notations}
\begin{itemize}
    \item $\vect{q} = (q_1, \ldots, q_n)$: generalized coordinates
    \item $\dot{\vect{q}} = (\dot{q_1}, \ldots, \dot{q_n})$: generalized velocity
    \item $\Lcal(t, q, \dot{q}) = T - V = \frac12 M \norm{\dot{q}}^2 - V(q) $: Lagrangian, where $M$ is the mass matrix, $T$ is kinetic energy and $V$ is potential energy
    \item $\vect{p}$: generalized momentum, i.e.,\ $\vect{p} = M \cdot \dot{\vect{q}}$
\end{itemize}


\section{Euler--Lagrange Equation}
Generalized version: given a funtion $\vect{y}(x)$ and its derivative: $\vect{y}'(x)$ and $f(x, \vect{y}, \vect{y}')$. Define the functional:
\begin{equation}
F[\vect{y}] = \int_a^b f(x, \vect{y}, \vect{y}') dx \label{eqn:functional}
\end{equation}
As the topic in calculus of variation, we want the functional $F[\vect{y}]$ to obtain its local min for some function $\vect{y}$. Perturbate $\vect{y}(x)$ with $\vect{y}(x) + \eps \vect{u}(x)$ for any function $\vect{u}$ and $\eps \in \R$ with small enough $\abs{\eps}$. and put \[g(\eps) = \int_a^b f(x, \vect{y + \eps u}, (\vect{y + \eps u})') dx\]
To make $F[\vect{y}]$ obtaining its local min for some $\vect{y}$ satisfying the initial conditions: $\vect{y}(a)$ and $\vect{y}(b)$ being fixed, i.e.,\ $\vect{u}(a) = \vect{u}(b) = 0$, it is equivalent to require $g'(0) = 0$, hence:
\begin{align*}
& g'(0) = \frac{d}{d\eps}\Big(\int_a^b f(x, \vect{y + \eps u}, (\vect{y + \eps u})') dx\Big)\\
& = \int_a^b \Big(\pd{f}{\vect{y}}\vect{u} + \pd{f}{\vect{y}'}\vect{u}' \Big)dx \quad \text{By total derivative}\\
& = \int_a^b \vect{u} \pd{f}{\vect{y}}  dx + \pd{f}{\vect{y}'}\vect{u} \vert_a^b - \int_a^b \vect{u} \frac{d}{dx}\big(\pd{f}{\vect{y}'}\big) dx \quad \text{integration by parts}\\
& = \int_a^b \vect{u} \Big(\pd{f}{\vect{y}} - \frac{d}{dx}\pd{f}{\vect{y}'}\Big) dx = 0 \quad \text{By initial condition:} ~ \vect{u}(a) = \vect{u}(b) = 0
\end{align*}
By the fundamental lemma of calculus of variation, since $\vect{u}(x)$ is arbitrary, hence we must have:
\begin{equation}
\pd{f}{\vect{y}} - \frac{d}{dx}\pd{f}{\vect{y}'} = 0 \label{eqn:Euler_Lagrangian}
\end{equation}
Equation (\ref{eqn:Euler_Lagrangian}) is called the \textbf{Euler--Lagrangian Equation} in calculus of variation, and for $\vect{y}$ satisfying Equation (\ref{eqn:Euler_Lagrangian}) is a sufficient condition for $F[\vect{y}]$ to have its local min.

In a dynamical system, with Lagrangian $\Lcal(t, \vect{q}, \dot{\vect{q}})$, we replace $f$ with $\Lcal$, $\vect{y}$ with generalized coordiantes $\vect{q}$, and $x$ with time $t$, we get the \textbf{Lagrangian Equation} for dynamical system:
\begin{equation}
\pd{\Lcal}{\vect{q}} - \frac{d}{dt}\pd{\Lcal}{\dot{\vect{q}}} = 0 \label{eqn:Lagrangian} 
\end{equation}

\section{Legendre Transform}



\section{Hamiltonian Equation}
Hamiltonian is the Legendre transform of Lagrangian, hence we have:
\begin{equation}
\Hcal(\vect{q}, \vect{p}) = \vect{p} \cdot \dot{\vect{q}} - \Lcal(t, \vect{q}, \dot{\vect{q}}) \label{eqn:Hamiltonian}
\end{equation}
Since $\vect{p}\cdot \dot{\vect{q}} = 2T$, so $\Hcal(\vect{q}, \vect{p}) = 2T - (T - V) = T + V$ and hence Hamiltonian can be interpreted as the total energy of a dynamical system.


We can derive Hamiltonian equations from Lagrange eqaution. Note that:\[\pd{\Lcal}{\dot{\vect{q}}} = M\dot{\vect{q}} = \dot{\vect{p}} \quad \text{and} \quad \pd{\Lcal}{\vect{q}} = \frac{d}{dt}\pd{\Lcal}{\dot{\vect{q}}} = \dot{\vect{p}}\]
so with (\ref{eqn:Hamiltonian}), we must have:
\begin{align}
& \pd{\Hcal}{\vect{p}} = \dot{\vect{q}} \label{eqn:Hamiltonian_equation1}\\
& \pd{\Hcal}{\vect{q}} = -\pd{\Lcal}{q} = - \dot{\vect{p}} \label{eqn:Hamiltonian_equation2}
\end{align}
(\ref{eqn:Hamiltonian_equation1}) and (\ref{eqn:Hamiltonian_equation2}) are called \textbf{Hamiltonian equations}

Or equivalently, we can take the total differentiation on both sides of (\ref{eqn:Hamiltonian}):
\begin{align*}
& d\Hcal = d(\vect{p}\cdot \dot{\vect{q}}) - d\Lcal(t, \vect{q}, \dot{\vect{q}})\\
& = \dot{\vect{q}}\cdot d\vect{p} + \vect{p} \cdot d\dot{\vect{q}} - \pd{\Lcal}{\vect{q}}d\vect{q} - \pd{\Lcal}{\dot{\vect{q}}}d\dot{\vect{q}} - \pd{\Lcal}{t}dt\\
& = \dot{\vect{q}}\cdot d\vect{p} - \pd{\Lcal}{\vect{q}}d\vect{q} - \pd{\Lcal}{t}dt\\
& = \pd{\Hcal}{\vect{p}}d\vect{p} + \pd{\Hcal}{\vect{q}} d\vect{q} + \pd{\Hcal}{t}dt
\end{align*}
Hence we can correspond the coefficients to get (\ref{eqn:Hamiltonian_equation1}) and (\ref{eqn:Hamiltonian_equation2}).

\section{Hamiltonian Monte Carlo}

\end{document}