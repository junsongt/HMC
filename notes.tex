\documentclass[hidelinks]{article}[12pt]
% \documentclass{article}
% \usepackage{multirow,makeidx,algorithmic,algorithm}
\usepackage{multirow,makeidx,algpseudocode,algorithm}
\usepackage{mathtools}
\usepackage{enumitem}
\usepackage{booktabs}
 
\usepackage{parskip}
\usepackage{setspace}
\usepackage{float}
\usepackage{adjustbox}
\usepackage{bbm}
\usepackage{tabularx}
\usepackage{subfigure}
\usepackage{amsmath,amssymb,amsfonts,amsthm,amsbsy,amstext,mathrsfs}
\usepackage{microtype}

% \usepackage{hyperref}
\usepackage[colorlinks=true, % turn on colored links
            linkcolor=cyan,% internal links (sections, etc.)
            citecolor=blue, % citation links
            urlcolor=blue  % URL links
           ]{hyperref}
\usepackage{url}
\usepackage{color}
\usepackage{xcolor}
\usepackage{graphicx} % Required for inserting images
\usepackage[utf8]{inputenc}
\usepackage{appendix}


%% reference
\usepackage[round]{natbib}
\bibliographystyle{abbrvnat}
% \usepackage{cite}
% \bibliographystyle{plainurl}
% \bibliographystyle{abbrv}
% \bibliographystyle{plain}
% \bibliographystyle{unsrt}


%% code in-text
\usepackage{listings}
\usepackage{xcolor}

\definecolor{codegreen}{rgb}{0,0.6,0}
\definecolor{codegray}{rgb}{0.5,0.5,0.5}
\definecolor{codepurple}{rgb}{0.58,0,0.82}
\definecolor{backcolour}{rgb}{0.95,0.95,0.92}

\lstdefinestyle{mystyle}{
    backgroundcolor=\color{backcolour},   
    commentstyle=\color{codegreen},
    keywordstyle=\color{magenta},
    numberstyle=\tiny\color{codegray},
    stringstyle=\color{codepurple},
    basicstyle=\ttfamily\footnotesize,
    breakatwhitespace=false,         
    breaklines=true,                 
    captionpos=b,                    
    keepspaces=true,                 
    numbers=left,                    
    numbersep=5pt,                  
    showspaces=false,                
    showstringspaces=false,
    showtabs=false,                  
    tabsize=2
}

\lstset{style=mystyle}


%% layout
\oddsidemargin 3mm
\evensidemargin 3mm
\topmargin -12mm
\textheight 660pt
\textwidth 450pt


% \newcounter{xxx}
% \setcounter{xxx}{0}
% \newcommand\XXX[1]{{\bf \em \addtocounter{xxx}{1} (\thexxx) [[#1]]}}
% \newcommand{\revise}[1]{{\color{red} #1}}


\floatstyle{ruled}
\newfloat{function}{htbp}{lof}% “lof” = file extension for list
\floatname{function}{Function}% caption prefix
\newcommand{\listoffunctions}{\listof{function}{List of Functions}}
  % DO NOT CHANGE
% ADD YOUR CUSTOM NOTATION HERE
\newcommand{\R}{\mathbb R}
\newcommand{\Z}{\mathbb Z}
\newcommand{\Q}{\mathbb Q}
\newcommand{\N}{\mathbb N}
\newcommand{\I}{\mathbb I}
\newcommand{\C}{\mathbb{C}}
\newcommand{\1}{\mathbbm{1}}
\newcommand{\E}{\mathbb E}
\newcommand{\Mcal}{{\cal M}}
\newcommand{\Ncal}{{\cal N}}
\newcommand{\Acal}{{\cal A}}
\newcommand{\Bcal}{{\cal B}}
\newcommand{\Fcal}{{\cal F}}
\newcommand{\Ecal}{{\cal E}}
\newcommand{\Gcal}{{\cal G}}
\newcommand{\Hcal}{{\cal H}}
\newcommand{\Lcal}{{\cal L}}
\newcommand{\Ocal}{{\cal O}}
\newcommand{\Scal}{{\cal S}}
\newcommand{\Ucal}{{\cal U}}
\newcommand{\Xcal}{{\cal X}}
\newcommand{\eps}{\varepsilon}
\renewcommand{\P}{\mathbb P}
\DeclareMathOperator{\Var}{Var}
\DeclareMathOperator{\Poi}{Poi}
\DeclareMathOperator{\Cov}{Cov}
\DeclareMathOperator{\Exp}{Exp}
\DeclareMathOperator{\Bin}{Bin}
\DeclareMathOperator{\Geom}{Geom}
\DeclareMathOperator{\Unif}{Unif}
\DeclareMathOperator{\Bernoulli}{Bernoulli}
\DeclareMathOperator{\Beta}{Beta}
\DeclareMathOperator{\Frac}{Frac}
\newcommand{\abs}[1]{\left|#1\right|}
\newcommand{\norm}[1]{\left\lVert#1\right\rVert}
\newcommand{\floor}[1]{\lfloor#1\rfloor}
\newcommand{\ceil}[1]{\lceil#1\rceil}
\newcommand{\ds}{\displaystyle}
\newcommand{\inv}[1]{#1^{-1}}
\newcommand{\vect}[1]{\boldsymbol{#1}}
\DeclareMathOperator*{\argmax}{arg\,max}
\DeclareMathOperator*{\argmin}{arg\,min}
\newcommand{\convdist}[0]{\overset{d}{\longrightarrow}}
\newcommand{\convprob}[0]{\overset{p}{\longrightarrow}}
\newcommand{\convas}[0]{\overset{a.s.}{\longrightarrow}}
\newcommand{\pd}[2]{\frac{\partial{#1}}{\partial{#2}}}
\newcommand{\pdd}[2]{\frac{\partial^2{#1}}{\partial{#2^2}}}
\newcommand{\Parameter}[2]{\Statex $\triangleright$ \texttt{#1}: #2}
% \newtheorem{definition}{Definition}[section]
% \newtheorem{theorem}{Theorem}[section]
% \newtheorem{corollary}{Corollary}[theorem]
% \newtheorem{lemma}{Lemma}[theorem]
% \newtheorem{proposition}[theorem]{Proposition}

\newtheorem{assumption}{Assumption}[section]
\newtheorem{definition}{Definition}[section]
\newtheorem{theorem}{Theorem}[section]
\newtheorem{corollary}{Corollary}[section]
\newtheorem{lemma}{Lemma}[section]
\newtheorem{proposition}{Proposition}[section]
\newtheorem*{remark}{Remark}

\renewcommand{\algorithmicrequire}{ \textbf{Input:}} %Use Input in the format of Algorithm
\renewcommand{\algorithmicensure}{ \textbf{Output:}} %UseOutput in the format of Algorithm


% %%
% % full alphabets of different styles
% %%

% % bf series
% \def\bfA{\mathbf{A}}
% \def\bfB{\mathbf{B}}
% \def\bfC{\mathbf{C}}
% \def\bfD{\mathbf{D}}
% \def\bfE{\mathbf{E}}
% \def\bfF{\mathbf{F}}
% \def\bfG{\mathbf{G}}
% \def\bfH{\mathbf{H}}
% \def\bfI{\mathbf{I}}
% \def\bfJ{\mathbf{J}}
% \def\bfK{\mathbf{K}}
% \def\bfL{\mathbf{L}}
% \def\bfM{\mathbf{M}}
% \def\bfN{\mathbf{N}}
% \def\bfO{\mathbf{O}}
% \def\bfP{\mathbf{P}}
% \def\bfQ{\mathbf{Q}}
% \def\bfR{\mathbf{R}}
% \def\bfS{\mathbf{S}}
% \def\bfT{\mathbf{T}}
% \def\bfU{\mathbf{U}}
% \def\bfV{\mathbf{V}}
% \def\bfW{\mathbf{W}}
% \def\bfX{\mathbf{X}}
% \def\bfY{\mathbf{Y}}
% \def\bfZ{\mathbf{Z}}

% % bb series
% \def\bbA{\mathbb{A}}
% \def\bbB{\mathbb{B}}
% \def\bbC{\mathbb{C}}
% \def\bbD{\mathbb{D}}
% \def\bbE{\mathbb{E}}
% \def\bbF{\mathbb{F}}
% \def\bbG{\mathbb{G}}
% \def\bbH{\mathbb{H}}
% \def\bbI{\mathbb{I}}
% \def\bbJ{\mathbb{J}}
% \def\bbK{\mathbb{K}}
% \def\bbL{\mathbb{L}}
% \def\bbM{\mathbb{M}}
% \def\bbN{\mathbb{N}}
% \def\bbO{\mathbb{O}}
% \def\bbP{\mathbb{P}}
% \def\bbQ{\mathbb{Q}}
% \def\bbR{\mathbb{R}}
% \def\bbS{\mathbb{S}}
% \def\bbT{\mathbb{T}}
% \def\bbU{\mathbb{U}}
% \def\bbV{\mathbb{V}}
% \def\bbW{\mathbb{W}}
% \def\bbX{\mathbb{X}}
% \def\bbY{\mathbb{Y}}
% \def\bbZ{\mathbb{Z}}

% % cal series
% \def\calA{\mathcal{A}}
% \def\calB{\mathcal{B}}
% \def\calC{\mathcal{C}}
% \def\calD{\mathcal{D}}
% \def\calE{\mathcal{E}}
% \def\calF{\mathcal{F}}
% \def\calG{\mathcal{G}}
% \def\calH{\mathcal{H}}
% \def\calI{\mathcal{I}}
% \def\calJ{\mathcal{J}}
% \def\calK{\mathcal{K}}
% \def\calL{\mathcal{L}}
% \def\calM{\mathcal{M}}
% \def\calN{\mathcal{N}}
% \def\calO{\mathcal{O}}
% \def\calP{\mathcal{P}}
% \def\calQ{\mathcal{Q}}
% \def\calR{\mathcal{R}}
% \def\calS{\mathcal{S}}
% \def\calT{\mathcal{T}}
% \def\calU{\mathcal{U}}
% \def\calV{\mathcal{V}}
% \def\calW{\mathcal{W}}
% \def\calX{\mathcal{X}}
% \def\calY{\mathcal{Y}}
% \def\calZ{\mathcal{Z}}


% %%%%%%%%%%%%%%%%%%%%%%%%%%%%%%%%%%%%%%%%%%%%%%%%%%%%%%%%%%
% % text short-cuts
% \def\iid{i.i.d.\ } %i.i.d.
% \def\ie{i.e.\ }
% \def\eg{e.g.\ }
% \def\Polya{P\'{o}lya\ }
% %%%%%%%%%%%%%%%%%%%%%%%%%%%%%%%%%%%%%%%%%%%%%%%%%%%%%%%%%%

% % set theory/measure theory
% \def\collection{\calC}
% \newcommand{\sigalg}[1]{\mathcal{#1}}
% \def\borel{\calB} %Borel sets
% \def\sigAlg{\sigalg{H}} %sigma-algebra
% \def\filtration{\calF} %filtration
% \newcommand{\msblSpace}[1]{(#1,\sigalg{#1})}
% \newcommand{\measSpace}[2][\mu]{(#2,\sigalg{#2},#1)}
% \newcommand{\borelSpace}[1]{(#1,\borel(#1))}
% \newcommand{\measFuncs}[1]{\sigalg{#1}^f}
% \newcommand{\pbblSpace}{(\Omega,\sigAlg)}
% \newcommand{\probSpace}[1][\bbP]{(\Omega,\sigAlg,#1)}

% \def\leb{\lambda}

% \def\finv{f^{-1}} % inverse
% \def\ginv{g^{-1}} % inverse

% % group theory
% \def\grp{\calG} %group

% % operators
% \def\P{\bbP} %fundamental probability
% \def\E{\bbE} %expectation
% % conditional expectation
% \DeclarePairedDelimiterX\bigCond[2]{[}{]}{#1 \;\delimsize\vert\; #2}
% \newcommand{\conditional}[3][]{\bbE_{#1}\bigCond*{#2}{#3}}
% \def\Law{\mathcal{L}} %law; this is by convention in the literature
% % \def\indicator{\mathds{1}} % indicator function
% % \def\1{{\mathbf 1}}
% % \def\indicator{\1}

% % binary relations
% \def\condind{{\perp\!\!\!\perp}} %independence/conditional independence
% \def\equdist{\stackrel{\text{\rm\tiny d}}{=}} %equal in distribution
% \def\equas{\stackrel{\text{\rm\tiny a.s.}}{=}} %euqal amost surely
% \def\simiid{\sim_{\mbox{\tiny iid}}} %sampled i.i.d

% % common vectors and matrices
% \def\onevec{\mathbf{1}}
% \def\iden{\mathbf{I}} % identity matrix
% \def\supp{\text{\rm supp}}

% % misc
% % floor and ceiling
% % \DeclarePairedDelimiter{\ceilpair}{\lceil}{\rceil}
% % \DeclarePairedDelimiter{\floor}{\lfloor}{\rfloor}
% \newcommand{\argdot}{{\,\vcenter{\hbox{\tiny$\bullet$}}\,}} %generic argument dot
%%%%%%%%%%%%%%%%%%%%%%%%%%%%%%%%%%%%%%%%%%%%%%%%%%%%%%%%%%


 




\title{\bf Notes on Hamiltonian Monte Carlo}
\author{Junsong Tang}
\date{\today}

\setlength\parindent{0pt}
\frenchspacing
\begin{document}
\maketitle

\section{Notations}
\begin{itemize}
    \item $\vect{q} = (q_1, \ldots, q_n)$: generalized coordinates
    \item $\dot{\vect{q}} = (\dot{q_1}, \ldots, \dot{q_n})$: generalized velocity
    \item $\Lcal(t, q, \dot{q}) = T - V = \frac12 M \norm{\dot{q}}^2 - V(q) $: Lagrangian, where $M$ is the mass matrix, $T$ is kinetic energy and $V$ is potential energy
    \item $\vect{p}$: generalized momentum, i.e.,\ $\vect{p} = M \cdot \dot{\vect{q}}$
\end{itemize}


\section{Euler--Lagrange Equation}
Generalized version: given a funtion $\vect{y}(x)$ and its derivative: $\vect{y}'(x)$ and $f(x, \vect{y}, \vect{y}')$. Define the functional:
\begin{equation}
F[\vect{y}] = \int_a^b f(x, \vect{y}, \vect{y}') dx \label{eqn:functional}
\end{equation}
As the topic in calculus of variation, we want the functional $F[\vect{y}]$ to obtain its local min for some function $\vect{y}$. Perturbate $\vect{y}(x)$ with $\vect{y}(x) + \eps \vect{u}(x)$ for any function $\vect{u}$ and $\eps \in \R$ with small enough $\abs{\eps}$. and put \[g(\eps) = \int_a^b f(x, \vect{y + \eps u}, (\vect{y + \eps u})') dx\]
To make $F[\vect{y}]$ obtaining its local min for some $\vect{y}$ satisfying the initial conditions: $\vect{y}(a)$ and $\vect{y}(b)$ being fixed, i.e.,\ $\vect{u}(a) = \vect{u}(b) = 0$, it is equivalent to require $g'(0) = 0$, hence:
\begin{align*}
& g'(0) = \frac{d}{d\eps}\Big(\int_a^b f(x, \vect{y + \eps u}, (\vect{y + \eps u})') dx\Big)\\
& = \int_a^b \Big(\pd{f}{\vect{y}}\vect{u} + \pd{f}{\vect{y}'}\vect{u}' \Big)dx \quad \text{By total derivative}\\
& = \int_a^b \vect{u} \pd{f}{\vect{y}}  dx + \pd{f}{\vect{y}'}\vect{u} \vert_a^b - \int_a^b \vect{u} \frac{d}{dx}\big(\pd{f}{\vect{y}'}\big) dx \quad \text{integration by parts}\\
& = \int_a^b \vect{u} \Big(\pd{f}{\vect{y}} - \frac{d}{dx}\pd{f}{\vect{y}'}\Big) dx = 0 \quad \text{By initial condition:} ~ \vect{u}(a) = \vect{u}(b) = 0
\end{align*}
By the fundamental lemma of calculus of variation, since $\vect{u}(x)$ is arbitrary, hence we must have:
\begin{equation}
\pd{f}{\vect{y}} - \frac{d}{dx}\pd{f}{\vect{y}'} = 0 \label{eqn:Euler_Lagrangian}
\end{equation}
Equation (\ref{eqn:Euler_Lagrangian}) is called the \textbf{Euler--Lagrangian Equation} in calculus of variation, and for $\vect{y}$ satisfying equation (\ref{eqn:Euler_Lagrangian}) is a sufficient condition for $F[\vect{y}]$ to have its local min.

In a dynamical system, with Lagrangian $\Lcal(t, \vect{q}, \dot{\vect{q}})$, we replace $f$ with $\Lcal$, $\vect{y}$ with generalized coordiantes $\vect{q}$, and $x$ with time $t$, we get the \textbf{Lagrangian Equation} for dynamical system:
\begin{equation}
\pd{\Lcal}{\vect{q}} - \frac{d}{dt}\pd{\Lcal}{\dot{\vect{q}}} = 0 \label{eqn:Lagrangian} 
\end{equation}

\section{Legendre Transform}
If $f: \Xcal \to \R$ is convex, then $\forall t \in X$, then $f(t) \geq pt + b$, where $p = f'(x)$ and $b = f(x) - px$. This implies that:
\[pt - f(t) \leq px -f(x), \forall t \in \Xcal\]
Then the one-dimensional Legendre transform is defined as: 
\[f^*(p) = px - f(x) = \sup\{pt - f(t) : t \in \Xcal\}\]

If $\Xcal \subset \R^n$, then the generalized Legendre tranform is defined as:
\[f^*(\vect{p}) = \vect{p} \cdot \vect{x} - f(\vect{x})\]where $\vect{p} \cdot \vect{x} = \vect{p}^{\top} \vect{x}$ representing the inner product. 

Note that $f^*$ is also a convex function, since given a $t \in \Xcal$, the map $p \mapsto pt -f(t)$ is linear, hence the supremum of the family of linear maps: $\{p \mapsto pt-f(t)\}_{t \in \Xcal}$ gives an envolope, which is $f^*$. On the other hand, one can show:
\[f(x) = \sup\{px - f^*(p) : p \in \Xcal^*\}\]
which means that $f(x)$ is the envolope of the family of linear maps: $\{x \mapsto px - f^*(p)\}_{p \in \Xcal^*}$. 



\section{Hamiltonian Equation}
Hamiltonian is the Legendre transform of Lagrangian, hence we have:
\begin{equation}
\Hcal(\vect{q}, \vect{p}) = \vect{p} \cdot \dot{\vect{q}} - \Lcal(t, \vect{q}, \dot{\vect{q}}) \label{eqn:Hamiltonian}
\end{equation}
Since $\vect{p}\cdot \dot{\vect{q}} = 2T$, so $\Hcal(\vect{q}, \vect{p}) = 2T - (T - V) = T + V$ and hence Hamiltonian can be interpreted as the total energy of a dynamical system.


We can derive Hamiltonian equations from Lagrange eqaution. Note that:\[\pd{\Lcal}{\dot{\vect{q}}} = M\cdot\dot{\vect{q}} = \vect{p} \quad \text{and} \quad \pd{\Lcal}{\vect{q}} = \frac{d}{dt}\pd{\Lcal}{\dot{\vect{q}}} = \dot{\vect{p}}\]
so with (\ref{eqn:Hamiltonian}), we must have:
\begin{align}
& \pd{\Hcal}{\vect{p}} = \dot{\vect{q}} \label{eqn:Hamiltonian_equation1}\\
& \pd{\Hcal}{\vect{q}} = -\pd{\Lcal}{q} = - \dot{\vect{p}} \label{eqn:Hamiltonian_equation2}
\end{align}
(\ref{eqn:Hamiltonian_equation1}) and (\ref{eqn:Hamiltonian_equation2}) are called \textbf{Hamiltonian equations}

Or equivalently, we can take the total differentiation on both sides of (\ref{eqn:Hamiltonian}):
\begin{align*}
& d\Hcal = d(\vect{p}\cdot \dot{\vect{q}}) - d\Lcal(t, \vect{q}, \dot{\vect{q}})\\
& = \dot{\vect{q}}\cdot d\vect{p} + \vect{p} \cdot d\dot{\vect{q}} - \pd{\Lcal}{\vect{q}}d\vect{q} - \pd{\Lcal}{\dot{\vect{q}}}d\dot{\vect{q}} - \pd{\Lcal}{t}dt\\
& = \dot{\vect{q}}\cdot d\vect{p} - \pd{\Lcal}{\vect{q}}d\vect{q} - \pd{\Lcal}{t}dt\\
& = \pd{\Hcal}{\vect{p}}d\vect{p} + \pd{\Hcal}{\vect{q}} d\vect{q} + \pd{\Hcal}{t}dt
\end{align*}
Hence we can correspond the coefficients to get (\ref{eqn:Hamiltonian_equation1}) and (\ref{eqn:Hamiltonian_equation2}).

\section{Hamiltonian Monte Carlo}

\end{document}